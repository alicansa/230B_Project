\documentclass[]{article}



\usepackage{graphicx,forloop,caption,subcaption,float,hyperref,arrayjob,listings,color,booktabs,mathtools}
\usepackage{pdfpages}
\usepackage[margin=1.2in]{geometry}
\usepackage{amsmath}
\usepackage{multirow}
%vhdl code
\definecolor{dkgreen}{rgb}{0,0.6,0}
\definecolor{gray}{rgb}{0.5,0.5,0.5}
\definecolor{mauve}{rgb}{0.58,0,0.82}

\DeclareMathOperator*{\argmin}{\arg\!\min}
\newcommand{\rom}[1]{\uppercase\expandafter{\romannumeral#1}}

\lstset{frame=tb,
  language=VHDL,
  aboveskip=3mm,
  belowskip=3mm,
  showstringspaces=false,
  columns=flexible,
  basicstyle={\small\ttfamily},
  numbers=none,
  numberstyle=\tiny\color{gray},
  keywordstyle=\color{blue},
  commentstyle=\color{dkgreen},
  stringstyle=\color{mauve},
  breaklines=true,
  breakatwhitespace=true
  tabsize=3
}

%matlab code
\lstset{frame=tb,
  language=Matlab,
  aboveskip=3mm,
  belowskip=3mm,
  showstringspaces=false,
  columns=flexible,
  basicstyle={\small\ttfamily},
  numbers=none,
  numberstyle=\tiny\color{gray},
  keywordstyle=\color{blue},
  commentstyle=\color{dkgreen},
  stringstyle=\color{mauve},
  breaklines=true,
  breakatwhitespace=true
  tabsize=3
}


% Title Page
\title{UCLA\\EE230B\\Digital Communication Design Project\\Step 2 Report}
\author{Alican Salor 404271991 \\  \href{mailto:alicansalor@ucla.edu}{alicansalor@ucla.edu} \\ \\
Darren Reis 804359840 \\
\href{mailto:darrer.r.reis@gmail.com}{darren.r.reis@gmail.com} }


\begin{document}
\maketitle

\newpage
\tableofcontents

\newpage


\section{System Setup}
\label{sec:setup}
The system is as shown in Figure below:

\begin{figure}[H]
\centering
\includegraphics[width=\textwidth]{step2.jpg}
\caption{Diagram of the setup used in step 2 of the project}
\end{figure}

In addition to the modeling blocks used previously, non coherent error was included in the model.  The $\delta\phi$ block represents a block that introduces phase or frequency offset on the carrier.  The effect of this is to rotate and blur the constellations and the SNR plots [Section~\ref{sec:qam16_phaseConst}].

The new block was modeled by two different functions.  For Phase offset, the signal was subjected to a constant phase bias, as described in Appendix~\ref{app:phase_offset}.  Similarly, the Frequency offsets were handled by introducing a first order phase term.  Recall, phase is related to frequency by $f(t) = \frac{1}{2 \pi} \phi^\prime(t)$.  Appendix~\ref{app:freq_offset} shows how this was implemented in the simulations.

With the new design worked out, the system was put through a similar set of tests on the four modulation schemes: BPSK, QPSK, 16-QAM, 64-QAM.  Each was subjected to phase offsets from $5\deg$ to $45\deg$ for a range of SNR levels.  Similarly, the modulated signals faced frequency offsets from 10 mHz to 10 Hz.  The system performance was determined by comparing a theoretical error rate to experimental bit error rate.  
\section{Derivation of Errors}
\label{sec:errors}
The following are theoretical formulas for bit error rate under phase error \cite{howald}.  The first is a simple derivation based on a $\phi$ error.  Refer to Figure~\ref{fig:bpskDraw}.

\begin{figure}[H]
\centering
\hspace*{-2cm}\includegraphics[width=0.4\textwidth]{bpskDraw.jpg}
\caption{The rotation of a BPSK constellation is based on the phase error, $\phi$ \label{fig:bpskDraw}}
\end{figure}

\begin{equation}
P_e^{\left(BPSK\right)} = Q\left(\frac{2cos\left(\phi\right)^{2}E_b}{N_0}\right)
\end{equation}

Next, the QPSK constellation error can be shown in Figure~\ref{fig:qpskDraw} to be:
\begin{equation}
P_e^{\left(QPSK \right)} = Q\left(\sqrt{\frac{2E_b}{N_0}}\sin\left(\frac{\pi}{4}-\phi\right)\right) + Q\left(\sqrt{\frac{2E_b}{N_0}}\sin\left(\frac{\pi}{4}+\phi\right)\right)
\end{equation}

\begin{figure}[H]
\centering
\hspace*{-2cm}\includegraphics[width=0.4\textwidth]{qpskDraw.jpg}
\caption{The rotation of a QPSK constellation is based on the phase error \label{fig:qpskDraw}}
\end{figure}

The QAM phase error derivations get to be complicated.  As shown in Figure~\ref{fig:qamDraw}, there are multiple region types.  As such, different symbols have different error probabilities.  Equation~\ref{eq:qam16} shows the average error probability for the entire constellation.  This equation turns out to be superfluous since very little phase error all-together ruins the transmission.  Refer to Section~\ref{sec:qam16_phase}.
\begin{align}
P_{e,\rom{1}}^{16-QAM} &= Q\left(\sqrt{\frac{4E_b}{5N_0}}3\sqrt{2}\cos\left(\frac{\pi}{4}+\phi)-2\right)\right) + Q\left(\sqrt{\frac{4E_b}{5N_0}}3\sqrt{2}sin\left(\frac{\pi}{4}+\phi)-2\right)\right) \\
P_{e,\rom{2}}^{16-QAM} &= Q\left(\sqrt{\frac{4E_b}{5N_0}}\left(\sqrt{10}\cos\left(\frac{\pi}{10}+\phi\right)-2\right)\right)  +  Q\left(\sqrt{\frac{4E_b}{5N_0}}\left(2-\sqrt{10}\sin\left(\frac{\pi}{10}+\phi)\right)\right)\right) \nonumber \\
&\quad {} + Q\left(\sqrt{\frac{8E_b}{N_0}}\left(\cos\left(\frac{\pi}{10}+\phi\right)\right)\right) \\
P_{e,\rom{3}}^{16-QAM} &= Q\left(\sqrt{\frac{8E_b}{N_0}}\cos\left(\frac{2\pi}{5}+\phi\right)\right)  +  Q\left(\sqrt{\frac{4E_b}{5N_0}}\left(2-\sqrt{10}\cos\left(\frac{2\pi}{5}+\phi \right)\right)\right) \nonumber \\
&\quad {} + Q\left(\sqrt{\frac{4E_b}{5N_0}}\left(\sqrt{10}\sin\left(\frac{2\pi}{5}+\phi\right)-2\right)\right)\\
P_{e,\rom{4}}^{16-QAM} &= Q\left(\sqrt{\frac{8E_b}{5N_0}}\cos\left(\frac{\pi}{4}+\phi\right)\right)  +  Q\left(\sqrt{\frac{8E_b}{5N_0}}\left(\sin\left(\frac{\pi}{4}+\phi \right)\right)\right) \nonumber \\
&\quad {} + Q\left(\sqrt{\frac{4E_b}{5N_0}}\left(2-\sqrt{2}\sin\left(\frac{\pi}{4}+\phi\right)\right)\right) + Q\left(\sqrt{\frac{4E_b}{5N_0}}\left(2-\sqrt{2}\cos\left(\frac{\pi}{4}+\phi\right)\right)\right) \\
\label{eq:qam16}
P_{e}^{16-QAM} &= \frac{1}{4}\left( 4P_{e,\rom{1}}^{\text{16-QAM}} + 4P_{e,\rom{2}}^{\text{16-QAM}} + 4P_{e,\rom{3}}^{\text{16-QAM}} + 4P_{e,\rom{4}}^{\text{16-QAM}} \right)
\end{align}

\begin{figure}[H]
\centering
\hspace*{-2cm}\includegraphics[width=0.4\textwidth]{qamDraw.jpg}
\caption{The rotation of a 16-QAM constellation is based on the phase error \label{fig:qamDraw}}
\end{figure}

\section{Step 2 Phase Offset Results}
\label{sec:results_po}
In the following sections, the probability of bit error versus the signal to noise ratio is shown on scatter plots for BPSK, QPSK, 16-QAM and 64-QAM constellations.  Notice that the constellations show rotation from the phase errors.  

\begin{figure}[H]
\centering
\hspace*{-2cm}\includegraphics[width=0.4\textwidth]{qpskError.png}
\caption{The rotation of a constellation is based on the phase error}
\end{figure}

\subsection{Probability of Error vs SNR plots for constellations with Phase Offset}
\begin{itemize}
\item Theoretical Symbol Error Rate curve as a function of the symbol SNR
\item The Symbol Error Rate curve from the simulation as a function of the received signal's SNR
\end{itemize}

As the phase error builds until $\frac{\pi}{4}$, the constellation becomes more and more shifted.  At some point, the clumps of symbols are far enough out of phase to be random, or equally likely right as wrong.  This happens for phase offsets of 45 degrees.
\subsubsection{BPSK with Phase Offset}
\label{sec:bpsk_phase}
\begin{figure}[H]
\centering
\hspace*{-2cm}\includegraphics[width=1.3\textwidth]{bpSNRpo1.jpg}
\caption{BPSK Theoretical and Experimental error rates versus different SNR levels at a phase offset of 5 degrees }
\end{figure}

\begin{figure}[H]
\centering
\hspace*{-2cm}\includegraphics[width=1.3\textwidth]{bpSNRpo2.jpg}
\caption{BPSK Theoretical and Experimental error rates versus different SNR levels at a phase offset of 10 degrees }
\end{figure}


\begin{figure}[H]
\centering
\hspace*{-2cm}\includegraphics[width=1.3\textwidth]{bpSNRpo3.jpg}
\caption{BPSK Theoretical and Experimental error rates versus different SNR levels at a phase offset of 20 degrees }
\end{figure}

\begin{figure}[H]
\centering
\hspace*{-2cm}\includegraphics[width=1.3\textwidth]{bpSNRpo4.jpg}
\caption{BPSK Theoretical and Experimental error rates versus different SNR levels at a phase offset of 45 degrees }
\end{figure}


\subsubsection{QPSK with Phase Offset}
\label{sec:qpsk_phase}
\begin{figure}[H]
\centering
\hspace*{-2cm}\includegraphics[width=1.3\textwidth]{qpSNRpo1.jpg}
\caption{QPSK Theoretical and Experimental error rates versus different SNR levels at a phase offset of 5 degrees }
\end{figure}

\begin{figure}[H]
\centering
\hspace*{-2cm}\includegraphics[width=1.3\textwidth]{qpSNRpo2.jpg}
\caption{QPSK Theoretical and Experimental error rates versus different SNR levels at a phase offset of 10 degrees }
\end{figure}

\begin{figure}[H]
\centering
\hspace*{-2cm}\includegraphics[width=1.3\textwidth]{qpSNRpo3.jpg}
\caption{QPSK Theoretical and Experimental error rates versus different SNR levels at a phase offset of 20 degrees }
\end{figure}

\begin{figure}[H]
\centering
\hspace*{-2cm}\includegraphics[width=1.3\textwidth]{qpSNRpo4.jpg}
\caption{QPSK Theoretical and Experimental error rates versus different SNR levels at a phase offset of 45 degrees }
\end{figure}


\subsubsection{16-QAM with Phase Offset}
\label{sec:qam16_phase}
\begin{figure}[H]
\centering
\hspace*{-2cm}\includegraphics[width=1.3\textwidth]{qam16SNRpo1.jpg}
\caption{16-QAM Theoretical and Experimental error rates versus different SNR levels at a phase offset of 5 degrees }
\end{figure}

\begin{figure}[H]
\centering
\hspace*{-2cm}\includegraphics[width=1.3\textwidth]{qam16SNRpo2.jpg}
\caption{16-QAM Theoretical and Experimental error rates versus different SNR levels at a phase offset of 10 degrees }
\end{figure}

\begin{figure}[H]
\centering
\hspace*{-2cm}\includegraphics[width=1.3\textwidth]{qam16SNRpo3.jpg}
\caption{16-QAM Theoretical and Experimental error rates versus different SNR levels at a phase offset of 20 degrees }
\end{figure}

\begin{figure}[H]
\centering
\hspace*{-2cm}\includegraphics[width=1.3\textwidth]{qam16SNRpo4.jpg}
\caption{16-QAM Theoretical and Experimental error rates versus different SNR levels at a phase offset of 45 degrees }
\end{figure}

\subsubsection{64-QAM with Phase Offset}
\label{qam64_phase}
\begin{figure}[H]
\centering
\hspace*{-2cm}\includegraphics[width=1.3\textwidth]{qam64SNRpo1.jpg}
\caption{64-QAM Theoretical and Experimental error rates versus different SNR levels at a phase offset of 5 degrees }
\end{figure}

\begin{figure}[H]
\centering
\hspace*{-2cm}\includegraphics[width=1.3\textwidth]{qam64SNRpo2.jpg}
\caption{64-QAM Theoretical and Experimental error rates versus different SNR levels at a phase offset of 10 degrees }
\end{figure}

\begin{figure}[H]
\centering
\hspace*{-2cm}\includegraphics[width=1.3\textwidth]{qam64SNRpo3.jpg}
\caption{64-QAM Theoretical and Experimental error rates versus different SNR levels at a phase offset of 5 degrees }
\end{figure}

\begin{figure}[H]
\centering
\hspace*{-2cm}\includegraphics[width=1.3\textwidth]{qam64SNRpo4.jpg}
\caption{64-QAM Theoretical and Experimental error rates versus different SNR levels at a phase offset of 10 degrees }
\end{figure}

\subsection{Constellation Plots for 16-QAM with Phase Offset}
\label{sec:qam16_phaseConst}
In the following sections scatter plots of  16-QAM constellation are plotted at symbol/input SNRs of 3 dB, 6dB, 10dB, 20dB and a phase offsets of 5 degree, 10 degree, 20 degree, 45 degree .

\begin{figure}[H]
\centering
\hspace*{-2cm}\includegraphics[width=1.3\textwidth]{qam16Constpo1.jpg}
\caption{The 16-QAM constellation plots for different levels of SNR at a phase offset of 5 degrees}
\end{figure}

\begin{figure}[H]
\centering
\hspace*{-2cm}\includegraphics[width=1.3\textwidth]{qam16Constpo2.jpg}
\caption{The 16-QAM constellation plots for different levels of SNR at a phase offset of 10 degrees}
\end{figure}

\begin{figure}[H]
\centering
\hspace*{-2cm}\includegraphics[width=1.3\textwidth]{qam16Constpo3.jpg}
\caption{The 16-QAM constellation plots for different levels of SNR at a phase offset of 20 degrees}
\end{figure}

\begin{figure}[H]
\centering
\hspace*{-2cm}\includegraphics[width=1.3\textwidth]{qam16Constpo4.jpg}
\caption{The 16-QAM constellation plots for different levels of SNR at a phase offset of 45 degrees}
\end{figure}
\newpage
\section{Step 2 Frequency Offset Results}
\label{sec:results_fo}
\subsection{Constellation Plots for BPSK and QPSK with Frequency Offset}
In the following sections scatter plots of  BPSK and QPSK constellations are plotted at symbol/input SNRs of 3 dB, 6dB, 10dB and 20dB and frequency offsets of 0.01 Hz, 0.1 Hz, 1 Hz, 10 Hz.

We expect the frequency offset to cause a blurring around the unit circle.  However since our symbol period is small to begin with, the dilation of the blob is hard to see except in the high SNR plots.  For 10 Hz offset, with SNR of 20 dB, there is a clear stretching of the symbols.    

\subsubsection{BPSK with Frequency Offset}
\label{sec:bpsk_freqConst}
\begin{figure}[H]
\centering
\hspace*{-2cm}\includegraphics[width=1.3\textwidth]{bpConstfo1.jpg}
\caption{The BPSK constellation plots for different levels of SNR at a frequency offset of 0.01 Hz}
\end{figure}

\begin{figure}[H]
\centering
\hspace*{-2cm}\includegraphics[width=1.3\textwidth]{bpConstfo2.jpg}
\caption{The BPSK constellation plots for different levels of SNR at a frequency offset of 0.1 Hz}
\end{figure}

\begin{figure}[H]
\centering
\hspace*{-2cm}\includegraphics[width=1.3\textwidth]{bpConstfo3.jpg}
\caption{The BPSK constellation plots for different levels of SNR at a frequency offset of 1 Hz}
\end{figure}

\begin{figure}[H]
\centering
\hspace*{-2cm}\includegraphics[width=1.3\textwidth]{bpConstfo4.jpg}
\caption{The BPSK constellation plots for different levels of SNR at a frequency offset of 10 Hz}
\end{figure}

\subsubsection{QPSK with Frequency Offset}
\begin{figure}[H]
\centering
\hspace*{-2cm}\includegraphics[width=1.3\textwidth]{qpConstfo1.jpg}
\caption{The QPSK constellation plots for different levels of SNR at a frequency offset of 0.01 Hz}
\end{figure}

\begin{figure}[H]
\centering
\hspace*{-2cm}\includegraphics[width=1.3\textwidth]{qpConstfo2.jpg}
\caption{The QPSK constellation plots for different levels of SNR at a frequency offset of 0.1 Hz}
\end{figure}

\begin{figure}[H]
\centering
\hspace*{-2cm}\includegraphics[width=1.3\textwidth]{qpConstfo3.jpg}
\caption{The QPSK constellation plots for different levels of SNR at a frequency offset of 1 Hz}
\end{figure}

\begin{figure}[H]
\centering
\hspace*{-2cm}\includegraphics[width=1.3\textwidth]{qpConstfo4.jpg}
\caption{The QPSK constellation plots for different levels of SNR at a frequency offset of 10 Hz \label{fig:qpsk_freq}}
\end{figure}


\newpage
\section{Conclusion}
\label{sec:conc}
This section of the project showed off the effects of carrier phase and frequency offsets.  This occurs in real systems when coherency is not maintained.  Synchronization between transmitter and receiver oscillator is not perfectly maintained.  The models showed that symbol error rates remained functional for small errors in phase and frequency.  As phase got to $\frac{\pi}{4}$ off, the constellations became the same as for proper $\frac{\pi}{4}$-QPSK [Section~\ref{sec:qam16_phaseConst}].  At this extreme, there was significant misclassification of symbols.  \\

The frequency errors did completely not ruin the transmissions [\ref{sec:results_fo}].  Because the symbol period was so brief, the accumulation of error for small frequency deviations was kept small.  As the frequency difference climbed, the effect became more and more evident.  The tell-tale sign of an offset in frequency is shown in plots such as Figure~\ref{fig:qpsk_freq}.  The blur along the unit circle demonstrates the frequency error.  The errors were catastrophic in the case of a 10 Hz frequency offset.  \\

From this project, the sensitivity of modulation schemes was analyzed.  Impercise phase and frequency can ruin a data system.  In subsequent phases of the project, remedies for these issues will be explored.

\appendix
\newpage
\bibliographystyle{plain}
\bibliography{step2}
\newpage
%% the \\ insures the section title is centered below the phrase: Appendix B
%\section{Project Assignment}
%\label{app:assign}
%\includepdf[pages={1-5}]{project_overview.pdf}
%\cleardoublepage
%\newpage


\section{Random Bit Sequence Generator}
\label{app:random_bit_generator}
\lstinputlisting{random_bit_generator.m}

\section{Bit to Symbol Mappers}
\label{app:bittosym}
\subsection{BPSK Modulation }
\label{app:bpsk_mod}
% To convert program (e.g. C++ Fortran, Matlab, LaTeX) listings to a
% form easily includable in a LaTeX document
%
% type lgrind -s to see options
% lgrind -llatex -i sample-paper.tex > sampleinputtex
% creates a file sampleinput.tex which can then be included into this
% document simply by uncommenting the next line
%\lgrindfile{testinput.tex}

\lstinputlisting{bpsk_mod.m}

\subsection{QPSK Modulation}
\label{app:qpsk_mod}
\lstinputlisting{qpsk_mod.m}

\subsection{16-QAM Modulation}
\label{app:qam_16_mod}

\lstinputlisting{QAM_16_mod.m}

\subsection{64-QAM Modulation }
\label{app:qam_64_mod}
\lstinputlisting{QAM_64_mod.m}

\newpage
\section{Square Root Raised Cosine Filter}
\label{app:sqrt_raised_cosine}
\lstinputlisting{sqrt_raised_cosine.m}


\section{Up Sampler}
\label{app:impulse_train}
\lstinputlisting{impulse_train.m}

\section{Additive Gaussian White Noise Channel}
\label{app:awgn_channel}
\lstinputlisting{awgn_complex_channel.m}

\newpage
\section{Sampler}
\label{app:sampler}
\lstinputlisting{sampler.m}


\section{Decision Blocks}
\label{app:dblocks}
\subsection{BPSK Demodulation}
\label{app:bpsk_demod}
\lstinputlisting{bpsk_demod.m}

\newpage
\subsection{QPSK Demodulation}
\label{app:qpsk_demod}
\lstinputlisting{qpsk_demod.m}

\subsection{16-QAM Demodulation}
\label{app:qam_16_demod}
\lstinputlisting{QAM_16_demod.m}

\newpage
\subsection{64-QAM Demodulation}
\label{app:qam_64_demod}
\lstinputlisting{QAM_64_demod.m}

\newpage
\section{Offsets}
\label{app:offsets}
\subsection{Carrier Phase Offset}
\label{app:phase_offset}
\lstinputlisting{phase_offset.m}

\newpage
\subsection{Carrier Frequency Offset}
\label{app:freq_offset}
\lstinputlisting{freq_offset.m}


\end{document}