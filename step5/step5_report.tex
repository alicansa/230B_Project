\documentclass[]{article}



\usepackage{graphicx,forloop,caption,subcaption,float,hyperref,listings,color,booktabs,mathtools}
\usepackage{pdfpages}
\usepackage{float}
\usepackage[margin=1.2in]{geometry}
\usepackage{amsmath}
\usepackage{amsthm}
\usepackage{amsfonts}
\usepackage{multirow}
%vhdl code
\definecolor{dkgreen}{rgb}{0,0.6,0}
\definecolor{gray}{rgb}{0.5,0.5,0.5}
\definecolor{mauve}{rgb}{0.58,0,0.82}

\DeclareMathOperator*{\argmin}{\arg\!\min}
\newcommand{\rom}[1]{\uppercase\expandafter{\romannumeral#1}}

% declare theorem definitions
\newtheorem{thm}{Condition}

\lstset{frame=tb,
  language=VHDL,
  aboveskip=3mm,
  belowskip=3mm,
  showstringspaces=false,
  columns=flexible,
  basicstyle={\small\ttfamily},
  numbers=none,
  numberstyle=\tiny\color{gray},
  keywordstyle=\color{blue},
  commentstyle=\color{dkgreen},
  stringstyle=\color{mauve},
  breaklines=true,
  breakatwhitespace=true
  tabsize=3
}


%matlab code
\lstset{frame=tb,
  language=Matlab,
  aboveskip=3mm,
  belowskip=3mm,
  showstringspaces=false,
  columns=flexible,
  basicstyle={\small\ttfamily},
  numbers=none,
  numberstyle=\tiny\color{gray},
  keywordstyle=\color{blue},
  commentstyle=\color{dkgreen},
  stringstyle=\color{mauve},
  breaklines=true,
  breakatwhitespace=true
  tabsize=3
}


% Title Page
\title{UCLA\\EE230B\\Digital Communication Design Project\\Step 5 Report}
\author{Alican Salor 404271991 \\  \href{mailto:alicansalor@ucla.edu}{alicansalor@ucla.edu} \\ \\
Darren Reis 804359840 \\
\href{mailto:darrer.r.reis@gmail.com}{darren.r.reis@gmail.com} }


\begin{document}
\maketitle

\newpage
\tableofcontents

\newpage
\section{Background}
\label{sec:background}
This step of the project deals with the effect of Inter Symbol Interference (ISI), when residual signal from symbols meddles the level of subsequent symbols.  This has previously not been modeled in the system because we have been considering an ideal scenario.  In reality, transmission over a channel has to deal with the finite bandwidth of the medium.  Because of the bandlimiting, where the response of the system is 0 above a limiting frequency, the symbols will interfere with one another. To deal with the dispersion, the Zero-ISI condition [\ref{thm:zero}] must be met.  A number of techniques can be utilized to accomplish what effectively amounts to canceling out delayed versions of symbols:

\begin{itemize}
\item Use $C^{-1}\left(f\right)$ to undo the channel
\item Use precoding
\item Use Nyquist's Pulse-Shaping Criterion and MLSE
\item Use an Equalizer
\end{itemize}
For this project, we use various types of equalizers to handle the ISI [Section~\ref{sec:equal}].  For this to be effective, the system channel medium must be known or estimated [Section~\ref{sec:estimate}].\\

\begin{figure}[b]
\centering
\includegraphics[width=.6\textwidth]{equalizer.png}
\caption{Generic Equalizer Filter to zero out the ISI\label{fig:equalizer}}
\end{figure}

\begin{thm}
\label{thm:zero}
Zero-ISI:
$$x\left(nT\right) = \left\{
\begin{array}{c c}
1 & \quad n=0 \\
0 & \quad \text{else}
\end{array} \right.$$
\end{thm}

\subsection{Estimation}
\label{sec:estimate}
Considering this system, where Table~\ref{tab:filtersummary} describes the variables and Table~\ref{tab:Paramsummary} describes the dimension parameters, the channel must be known before anything else. \\

To do channel estimation, a known sequence is sent through the system and error on the signal at output is measured.  That is, the input ($r$) and output ($y$) of the filter are known, and the tap weights ($f$) are to be determined - we can look at the impulse response of the unknown channel facing the known input.  Once the channel is understood, we can use an assortment of metrics to perform equalization.

\section{Equalization}
\label{sec:equal}
An equalizer is a filter that zeros out the ISI in the end-to-end system.  It can be preset to handle the channel, or can adapt to the time-varying nature of a channel.  In the latter case, the equalizer parameter are adjusted on the fly by periodic transmission of a known sequence to re-estimate the channel.  In either case, the equalizer is a filter whose frequency response counteracts the system model such that Condition~\ref{thm:zero} is met. 

\begin{table}[H]
\begin{center}
\begin{tabular}{|c|c|c|c|}
\hline Variable & Meaning & Dimensions \\
\hline \hline
$\mathbf{I}$ & Symbol Source & $ m \times 1$ \\ \hline
$\mathbf{s}$ & Source & $m\times 1 $\\ \hline
$\mathbf{r}$ & Received Signal & $m\times 1$ \\ \hline
$R$ & Channel Response Matrix & $p\times n$ \\ \hline
$\mathbf{f}$ & Tap Line / Impulse Response & $n\times 1 $ \\ \hline
$\mathbf{y}$ & Equalizer Output & $ m\times 1 $ \\ \hline
 $\mathbf{e}$ & Training Error & $ m\times 1 $ \\ \hline
\end{tabular}
\caption{Summary of Signal Variables} \label{tab:filtersummary}
\end{center}
\end{table}

\begin{table}[b]
\begin{center}
\begin{tabular}{|c|c|}
\hline Parameter & Meaning \\
\hline \hline
$m$ & Signal Length \\ \hline
$N$ & Channel Filter Order \\ \hline
$n$ & Equalizer Filter Order \\ \hline
$p$ & Training Sequence Length \\ \hline
\end{tabular}
\caption{Summary of Parameters} \label{tab:Paramsummary}
\end{center}
\end{table}

The FIR form of the equalizer can then be written as Equation~\ref{eq:equalizerVector} and Equation~\ref{eq:equalizerMatrix}.  The compact form of this relation uses a matrix equation where the filter is expressed as a Toeplitz matrix.  This neat fact allows us to use the power of linear algebra to solve for the zero forcing channel.  
  
 
\begin{figure}[H]
\centering
\includegraphics[width=.6\textwidth]{tapEqualizer.png}
\caption{Tapped Delay Line Represenation\label{fig:tap}}
\end{figure}

\begin{equation}
\label{eq:equalizer}
y\left[k\right] = \sum_{j=0}^n f_jr\left[k-j\right]
\end{equation}

The direct form of the FIR equalizor is shown in Figure~\ref{fig:tap}.  This is a subblock diagram view of the equalizer filter.  The transfer function can be gathered from Equation~\ref{eq:equalizer}.  The objective of this filter is to counter the system channel.  \\

\begin{equation}
\label{eq:equalizerVector}
\left[ \begin{array}{c}
 y \left[n+1\right] \\
 y \left[n+2\right] \\
 y \left[n+3\right] \\
\vdots  \\
y\left[ p \right] \end{array} \right] = 
\begin{bmatrix} 
r \left[ n+1\right]  & r[n] & \cdots & r\left[ 1 \right] \\ 
r \left[ n+2\right]  & r[n+1] & \cdots & r\left[ 2 \right] \\ 
r \left[ n+2\right]  & r[n+2] & \cdots & r\left[ 3 \right] \\ 
\vdots & \vdots & & \vdots \\
r \left[p \right] & r\left[ p-1 \right] & \cdots & r\left[ p-n \right]
\end{bmatrix}
 \left[ \begin{array}{c} f_0 \\ f_1 \\ f_2 \\ \vdots \\ f_n \end{array} \right]
\end{equation}

\begin{equation}
\label{eq:equalizerMatrix}
\mathbf{y} = R\mathbf{f}
\end{equation}
What we want is to force the channel to zero for all other symbols other than the present one.  As an aside, because there is delay in the system, the intuitive sense of causality is blurred. That is, forward symbols from the present moment can actually cause interference to the present symbol. \\

\subsection{Zero Forcing Equalizer}
\label{sec:zf}
The first equalizer we considered was a Zero-Forcing (ZF) architecture.  To meet the goal of satisfying Condition~\ref{thm:zero}, the weight vector, $\mathbf{f}$, must perfectly negate all taps except the present-time one.  The present received symbol is a column within the $R$ matrix: $\mathbf{r}[i]$, often chosen to be the center one.  
$$ \mathbf{f}^{\ast}R = \mathbf{w}[i]^{\top} $$
Here, we used  $\mathbf{w} [i] \in \mathbb{R}^n$ to represent the i$^{th}$ Standard basis vector, or a column of zeros except in row $i$.  In this setting, we can find the tap weight vector to accomplish this by Equation~\ref{eq:zf}.  \\

A common metric for the price of Zero Forcing is the \emph{noise enhancement}.  Because the equalizer is amplifying the system frequency response in the channel-attenuated zones in order to maintain a uniformly flat character, the noise from the channel gets similarly amplified by the equalizer.  This is the cost of knocking out ISI.  For ZF, this is a factor by which the system SNR must be scaled by in order to maintain equivalent performance after the equalizer as compared to the output after the matched filter in a no-ISI setting.  It can lead to cases where there is infinite noise PSD after the equalizer.

\begin{equation}
\label{eq:zf} 
\mathbf{f}_{ZF} = R \left(R^{\ast}R \right)^{-1} \mathbf{w}
\end{equation}

\subsection{Mean Square Error Equalizer}
\label{sec:optimal}
Because a perfect ZF equalizer requires explicit knowledge of the channel and has the problem of noise enhancement, other observation-based equalization techniques are more common.  In order to find an optimal equalizer weighting function from observed data, we need to define a cost function, $J(\mathbf{f})$, as a metric to minimize.  In this way, we can formulate an optimal estimator in the sense of cost.  Here, we look at the Mean Square Error.  Mean Square Error can be interpreted as the deviation of the estimate from the truth, squared [Equation~\ref{eq:mse}].  This formulation uses the $R$ matrix from before and defines the estimate error as $\mathbf{e} = \mathbf{s} - \mathbf{y}$.  This setting is well studied and the optimal weighting vector, $\mathbf{f}_{MSE}$ is shown in Equation~\ref{eq:optimal}\footnote{Note that this setting assumes $R$ has more rows than columns and that such an inverse exists}.
\begin{equation}
\label{eq:mse} 
J_{MSE} \left( \mathbf{f}\right) = \mathbb{E} \left[ \left(\mathbf{s} - R \mathbf{f} \right)^2 \right]
\end{equation}

\begin{equation}
\label{eq:optimal}
\mathbf{f}_{MSE} = \left(R^{\ast}R\right)^{-1}R^{\ast}\mathbf{s}
\end{equation}

When such a weighting vector is used as the equalizer taps, the output reaches the minimum mean square error metric  [Equation~\ref{eq:mmse}].   It is interesting to note, the first term in this minimal cost function is the variance of the transmitted signal and the second term is a bias term.  Also realize the MSE equalizer is not a perfect tool either: there is no gaurentee that all ISI is removed.
\begin{equation}
\label{eq:mmse}
J_{MMSE} =  \sigma_{\mathbf{s}}^{2} - \mathbb{E} \left[ \mathbf{s}^{\ast} \mathbf{f}_{MSE} \right]
\end{equation}

\subsection{Decision-Feedback Equalizer}
\label{sec:dfe}
Another alternative is to use a feedback loop to handle the ISI.  A Decision-Feedback Equalizer detects symbols and uses a feedback loop to eliminate intersymbol interference from each symbol prior to detecting any following symbols.  This brings the issue of noise enhancement back into play.  As such, we can use MSE techniques together with feedback to focus on balancing the need to eliminate ISI and the danger of noise enhancement.  To do this, we use a feedforward correlator and a feedback loop to clean up the ISI.  Figure~\ref{fig:dfe} shows the block diagram for this type of equalizer. 

\begin{figure}[H]
\centering
\includegraphics[width=.6\textwidth]{mse_dfe.png}
\caption{MSE Decision Feedback Equalizer\label{fig:dfe}}
\end{figure}

\newpage
\section{System}
\label{sec:system}
The system simulation model is shown in Figure~\ref{fig:step5}.  As from Step 1, randomly generated bits [Appendix~\ref{app:random_bit_generator}] are converted into symbols [\ref{app:bittosym}] and then upsampled by adding in zeros [\ref{app:impulse_train}].  The result is then is run through a Square Root Raised Cosine (SRRC) pulse shape filter [\ref{app:sqrt_raised_cosine}].  The use of the raised cosine shape no longer, in itself, satisfies Condition~\ref{thm:zero}.  Thus we need to do equalization.
\begin{figure}[H]
\centering
\includegraphics[width=.6\textwidth]{step5.png}
\caption{Block Diagram of Step 5 System Setup\label{fig:step5}}
\end{figure}
For this setting, we assume knowledge of the channel response. The following bandlimited channel responses are used in this step of the project:

\begin{itemize}
\item $h_1(t) = 1\delta(t) - 0.25\delta(t - T_{sym})  $
\item $h_2(t) = 1\delta(t) - 0.25\delta(t - T_{sym})   + 0.125\delta(t - 2T_{sym}) $
\item $h_3(t) = 0.1\delta(t + T_{sym}) +1\delta(t - T_{sym}) - 0.25\delta(t - T_{sym})   $
\end{itemize} 

Now that we have a channel response that is non-ideal, we need to do equalization - essentially convolving the sampled signal with the inverse of the channel response as shown below: 

\begin{equation}
\label{eq:channel}
y\left[k\right] = \sum_{j=0}^N c[j]r\left[k-j\right]
\end{equation}
\\
In order to equalize the effects of the channels given above, three different equalizers were used: 

\begin{itemize}
\item ZF equalizer
\item MMSE equalizer
\item MMSE-DFE equalizer

with each one having
	\begin{itemize}
	\item 3 taps for $h_1(t)$
	\item 5 taps for $h_2(t)$
	\item 3 taps for $h_3(t)$
	\end{itemize}

\end{itemize}


The end of the simulation model is identical to the process in Step 1: a matched filter to the SRRC picks out the symbols from the noisy received signal.  Afterwards, a sampler recovers [Appendix~\ref{app:sampler}] the symbols before equalization of the channel. After that a demodulator converts the symbols back into bits [\ref{app:dblocks}].  



\section{Step 5 Results}
\label{sec:results}
In the following sections, the results of the simulations of the different modulation schemes are shown. 

\subsection{Probability Error Rate Comparison}
\label{sec:compare}

\subsubsection{BPSK}

\begin{figure}[H]
\centering
\includegraphics[width=0.7\textwidth]{bpSNR1.jpg}
\caption{Comparison of BER for BPSK system with different equalizers under channel response $h_1(t)$}
\end{figure}

\begin{figure}[H]
\centering
\includegraphics[width=0.7\textwidth]{bpSNR2.jpg}
\caption{Comparison of BER for BPSK system with different equalizers under channel response $h_2(t)$}
\end{figure}

\begin{figure}[H]
\centering
\includegraphics[width=0.7\textwidth]{bpSNR3.jpg}
\caption{Comparison of BER for BPSK system with different equalizers under channel response $h_3(t)$}
\end{figure}

\subsubsection{QPSK}

\begin{figure}[H]
\centering
\includegraphics[width=0.7\textwidth]{qpSNR1.jpg}
\caption{Comparison of SER for QPSK system with different equalizers under channel response $h_1(t)$}
\end{figure}

\begin{figure}[H]
\centering
\includegraphics[width=0.7\textwidth]{qpSNR2.jpg}
\caption{Comparison of SER for QPSK system with different equalizers under channel response $h_2(t)$}
\end{figure}

\begin{figure}[H]
\centering
\includegraphics[width=0.7\textwidth]{qpSNR3.jpg}
\caption{Comparison of SER for QPSK system with different equalizers under channel response $h_3(t)$}
\end{figure}

\newpage
\subsection{Constellation Comparison}
\label{sec:constCompare}

\subsubsection{BPSK with No Equalization}

\begin{figure}[H]
\centering
\includegraphics[width=0.7\textwidth]{bpConst1.jpg}
\caption{Constellation plot for a BPSK modulated signal with no equalization under channel reponse $h_1(t)$}
\end{figure}

\begin{figure}[H]
\centering
\includegraphics[width=0.7\textwidth]{bpConst2.jpg}
\caption{Constellation plot for a BPSK modulated signal with no equalization under channel reponse $h_2(t)$}
\end{figure}

\begin{figure}[H]
\centering
\includegraphics[width=0.7\textwidth]{bpConst3.jpg}
\caption{Constellation plot for a BPSK modulated signal with no equalization under channel reponse $h_3(t)$}
\end{figure}

\subsubsection{BPSK with ZF Equalization}

\begin{figure}[H]
\centering
\includegraphics[width=0.7\textwidth]{bpConstZF1.jpg}
\caption{Constellation plot for a BPSK modulated signal with ZF equalization under channel reponse $h_1(t)$}
\end{figure}

\begin{figure}[H]
\centering
\includegraphics[width=0.7\textwidth]{bpConstZF2.jpg}
\caption{Constellation plot for a BPSK modulated signal with ZF equalization under channel reponse $h_2(t)$}
\end{figure}

\begin{figure}[H]
\centering
\includegraphics[width=0.7\textwidth]{bpConstZF3.jpg}
\caption{Constellation plot for a BPSK modulated signal with ZF equalization under channel reponse $h_3(t)$}
\end{figure}

\subsubsection{BPSK with MMSE Equalization}

\begin{figure}[H]
\centering
\includegraphics[width=0.7\textwidth]{bpConstMMSE1.jpg}
\caption{Constellation plot for a BPSK modulated signal with MMSE equalization under channel reponse $h_1(t)$}
\end{figure}

\begin{figure}[H]
\centering
\includegraphics[width=0.7\textwidth]{bpConstMMSE2.jpg}
\caption{Constellation plot for a BPSK modulated signal with MMSE equalization under channel reponse $h_2(t)$}
\end{figure}

\begin{figure}[H]
\centering
\includegraphics[width=0.7\textwidth]{qpConstMMSE3.jpg}
\caption{Constellation plot for a BPSK modulated signal with MMSE equalization under channel reponse $h_3(t)$}
\end{figure}

\subsubsection{BPSK with MMSE-DFE Equalization}

\begin{figure}[H]
\centering
\includegraphics[width=0.7\textwidth]{bpConstMMSEDFE1.jpg}
\caption{Constellation plot for a BPSK modulated signal with MMSE-DFE equalization under channel reponse $h_1(t)$}
\end{figure}

\begin{figure}[H]
\centering
\includegraphics[width=0.7\textwidth]{bpConstMMSEDFE2.jpg}
\caption{Constellation plot for a BPSK modulated signal with MMSE-DFE equalization under channel reponse $h_2(t)$}
\end{figure}

\begin{figure}[H]
\centering
\includegraphics[width=0.7\textwidth]{bpConstMMSEDFE3.jpg}
\caption{Constellation plot for a BPSK modulated signal with MMSE-DFE equalization under channel reponse $h_3(t)$}
\end{figure}


\subsubsection{QPSK with No Equalization}

\begin{figure}[H]
\centering
\includegraphics[width=0.7\textwidth]{qpConst1.jpg}
\caption{Constellation plot for a QPSK modulated signal with no equalization under channel reponse $h_1(t)$}
\end{figure}

\begin{figure}[H]
\centering
\includegraphics[width=0.7\textwidth]{qpConst2.jpg}
\caption{Constellation plot for a QPSK modulated signal with no equalization under channel reponse $h_2(t)$}
\end{figure}

\begin{figure}[H]
\centering
\includegraphics[width=0.7\textwidth]{qpConst3.jpg}
\caption{Constellation plot for a QPSK modulated signal with no equalization under channel reponse $h_3(t)$}
\end{figure}

\subsubsection{QPSK with ZF Equalization}

\begin{figure}[H]
\centering
\includegraphics[width=0.7\textwidth]{qpConstZF1.jpg}
\caption{Constellation plot for a QPSK modulated signal with ZF equalization under channel reponse $h_1(t)$}
\end{figure}

\begin{figure}[H]
\centering
\includegraphics[width=0.7\textwidth]{qpConstZF2.jpg}
\caption{Constellation plot for a QPSK modulated signal with ZF equalization under channel reponse $h_2(t)$}
\end{figure}

\begin{figure}[H]
\centering
\includegraphics[width=0.7\textwidth]{qpConstZF3.jpg}
\caption{Constellation plot for a QPSK modulated signal with ZF equalization under channel reponse $h_3(t)$}
\end{figure}

\subsubsection{BPSK with MMSE Equalization}

\begin{figure}[H]
\centering
\includegraphics[width=0.7\textwidth]{qpConstMMSE1.jpg}
\caption{Constellation plot for a QPSK modulated signal with MMSE equalization under channel reponse $h_1(t)$}
\end{figure}

\begin{figure}[H]
\centering
\includegraphics[width=0.7\textwidth]{qpConstMMSE2.jpg}
\caption{Constellation plot for a QPSK modulated signal with MMSE equalization under channel reponse $h_2(t)$}
\end{figure}

\begin{figure}[H]
\centering
\includegraphics[width=0.7\textwidth]{qpConstMMSE3.jpg}
\caption{Constellation plot for a QPSK modulated signal with MMSE equalization under channel reponse $h_3(t)$}
\end{figure}

\subsubsection{QPSK with MMSE-DFE Equalization}

\begin{figure}[H]
\centering
\includegraphics[width=0.7\textwidth]{qpConstDFEMMSE1.jpg}
\caption{Constellation plot for a QPSK modulated signal with MMSE-DFE equalization under channel reponse $h_1(t)$}
\end{figure}

\begin{figure}[H]
\centering
\includegraphics[width=0.7\textwidth]{qpConstDFEMMSE2.jpg}
\caption{Constellation plot for a QPSK modulated signal with MMSE-DFE equalization under channel reponse $h_2(t)$}
\end{figure}

\begin{figure}[H]
\centering
\includegraphics[width=0.7\textwidth]{qpConstDFEMMSE3.jpg}
\caption{Constellation plot for a QPSK modulated signal with MMSE-DFE equalization under channel reponse $h_3(t)$}
\end{figure}


\newpage
\section{Conclusion}
\label{sec:conc}

In this step of the project we have dealt with the effect of Inter Symbol Interference (ISI), caused by the transmission of signals over a channel with the finite bandwidth. Because of the bandlimiting, where the response of the system is 0 above a limiting frequency, the symbols will interfere with one another. To deal with the dispersion, the Zero-ISI condition [\ref{thm:zero}] must be met. 

Equalization of the channel is chosen from a variety of techniques to accomplish the cancellation of delayed versions of the symbols. 

The equalizers tried out in this step, to achieve delayed symbol cancellation are:

\begin{itemize}
\item ZF equalizer
\item MMSE equalizer
\item MMSE-DFE equalizer

with each one having
	\begin{itemize}
	\item 3 taps for $h_1(t)$
	\item 5 taps for $h_2(t)$
	\item 3 taps for $h_3(t)$
	\end{itemize}
\end{itemize}

The results of the experiment are given in the previous section. From the BER and constellation plots obtained, it is clear each equalizer sufficiently cancels out interference of the delayed symbols. The BER of the system with QPSK and BPSK modulations are almost the same for theoretical results as when the channel is equalized. 

As expected, the best performing equalizer is MMSE-DFE.  Zero Forcing equalization performs worse than Minimum Mean Square Error equalization because the effect of AWGN noise is considered in the latter and not in ZF.  Without being penalized for noise amplification, Zero Forcing equalization has issues from noise enhancement. 

When the equalizer has a feedback loop, the noise error enhancement issue is addressed. This balances the need to eliminate ISI and the danger of noise enhancement. Therefore we see the best results with MMSE-DFE equalizer.

Still, by using additional taps, ZF and MMSE equalization can achieve the same BER results as the feedback method.  A feedback loop increases the complexity of the overall system and may not always be worthwhile. This is a design trade-off an engineer has to decide on when choosing between more memory or the complexity of a feedback loop.

\appendix
\newpage
\bibliographystyle{plain}
\bibliography{step4}
\newpage
%% the \\ insures the section title is centered below the phrase: Appendix B
%\section{Project Assignment}
%\label{app:assign}
%\includepdf[pages={1-5}]{project_overview.pdf}
%\cleardoublepage
%\newpage

\section{Random Bit Sequence Generator}
\label{app:random_bit_generator}
\lstinputlisting{random_bit_generator.m}

\section{Bit to Symbol Mapper}
\label{app:bittosym}
\subsection{BPSK Modulation}
\label{app:bpsk_mod}
\lstinputlisting{bpsk_mod.m}

\subsection{QPSK Modulation}
\label{app:qpsk_mod}
\lstinputlisting{qpsk_mod.m}

\section{Up Sampler}
\label{app:impulse_train}
\lstinputlisting{impulse_train.m}

\section{Square Root Raised Cosine Filter}
\label{app:sqrt_raised_cosine}
\lstinputlisting{sqrt_raised_cosine.m}

\section{Channel Models}
\subsection{Ideal AWGN Channel}
\label{app:awgn_channel}
\lstinputlisting{awgn_channel.m}

\subsection{Bandlimited Channel}
\label{app:bandlimited}
\lstinputlisting{bandlimited_channel.m}

\section{Sampler}
\label{app:sampler}
\lstinputlisting{sampler.m}

\section{Decision Block}
\label{app:dblocks}
\subsection{BPSK Demodulation}
\label{app:bpsk_demod}
\lstinputlisting{bpsk_demod.m}

\subsection{QPSK Demodulation}
\label{app:qpsk_demod}
\lstinputlisting{qpsk_demod.m}

\section{Equalizers}
\label{app:equal}

\subsection{Zero Forcing Equalizer}
\label{app:zf}
\lstinputlisting{ZFEqualizer.m}

\subsection{Mean Square Error Equalizer}
\label{app:mse}
\lstinputlisting{MMSE_Equalizer.m}

\subsection{Mean Square Error Decision Feedback Equalizer}
\label{app:dfe}
\lstinputlisting{MMSE_Equalizer_Train.m}

\section{Simulations}
\subsection{BPSK Simulation}
\lstinputlisting{step5_sim_bpsk.m}

\subsection{QPSK Simulation}
\lstinputlisting{step5_sim_qpsk.m}

\end{document}
